\documentclass{beamer}
\usepackage[utf8]{inputenc}
\usepackage[croatian]{babel}
\usepackage{helvet}

	\usetheme{Boadilla}
	\usecolortheme{crane}

	\title[TikZ paket]{Izrada grafike koristeći TikZ paket}
	\subtitle{Računalne vještine}
	\author{V. Salamon, K.Pavlović, D.Silić}
	\Institute{Riteh}
	\date

\begin{document}

	\begin{frame}
	\titlepage
	\end{frame}

	\begin{frame}
	\frametitle{Sažetak}
		\begin{itemize}
			\item Vektorska grafika
			\item PGF/TikZ
			\item TikZ
			\item Korištenje TIkZ-a
		\end{itemize}	
	\end{frame}

	\begin{frame}
	\frametitle{Vektorska grafika}
		\begin{itemize}
			\item Koristi dvodimenzionalne poligone za reprodukciju slika u računalnoj grafici
			\item Svaka točka ima određenu poziciju na x i y osi ravnine te određuje put staze
			\item Svakoj stazi je moguće dodijeliti različite atribute poput: boje, oblika, krivulje, debljine i punjenja
			\item Primjer: razmatramo krug radijusa r; podatci koji su potrebni računarskom programu za iscrtavanje tog kruga su:
			\begin{enumerate}
			\item radijus r
			\item koordinatna pozicija središnje točke kruga
			\item stil i boja linije
			\item stil i boja punjenja objekta
			\end{enumerate}	
		\end{itemize}
	\end{frame}

	\begin{frame}
	\frametitle{Prednosti vektorske grafike}
	\begin{itemize}
		\item Prednost vektorske grafike naprema rasterskoj su:
		\begin{enumerate}
		\item manja količina informacija omogućava manju veličinu datoteke
		\item  sve su informacije zapamćene i mogu se kasnije mijenjati; 
    	to znači da kasnije možemo izmjenjivati svojstva slike
    	bez smanjenja kvalitete crteža kao kod rasterkse slike
		\end{enumerate}	
		\end{itemize}
	\end{frame}

	\begin{frame}
	\frametitle{PGF/TikZ}
		\begin{itemize}
		\item TikZ - najsloženiji i najmoćniji alat za stvaranje grafičkih elemenata u LaTeX-u
		\item PGF/TikZ - dva jezika koji rade u paru, služe nam za izradu vektorske grafike pomoću geometrijskog odnosno algebarskog opisa
		\item PGF („Prijenosni grafički format”) - osnovni sloj koji pruža korisniku osnovne naredbe za izradu grafike 
  		\item TikZ - sučelje sa posebnom sintaksom koja olakšava uporabu PGF-a
  		\item PGF je jezik niže razine, dok je TikZ skup makronaredbi na višoj razini koje koriste PGF
  		\end{itemize}	
	\end{frame}	

	\begin{frame}
	\frametitle{TikZ}
		\begin{itemize}
		\item „TikZ ist kein Zeichenprogramm” (TikZ nije program za crtanje)
		\item Dobivamo sve prednosti korištenja TeX-a za grafiku poput:
			\begin{enumerate}
			\item brzog stvaranja jednostavne grafike
			\item korištenje makronaredbi
 			\item precizno pozicioniranje
			\end{enumerate}
		\item Također i nedostatke:	
			\begin{enumerate}
			\item strma krivulja učenja
			\item nema WYSIWYG pristupa rada
 			\item male promjene zahtijevaju dugo vrijeme ponovnog prevođenja
			\end{enumerate}
  		\end{itemize}	
	\end{frame}

	\begin{frame}
	\frametitle{Usporedba sa ostalim grafičkim paketima}
		\begin{itemize}
			\item Usporedba TikZ-a u odnosu na druge pakete:
			\begin{itemize}
			\item standardna LaTeX-ova {picture} okolina nam omogućava izradu jednostavne grafike, ali ništa više od toga - rezultat portabilnosti {picture} okoline. Ne radi s pdftexom niti s bilo kojim drugim upravljačkim programom koji ne proizvodi ništa drugo osim PostScript koda.
  			\item Xypic je nešto stariji paket za izradu grafike. Nešto teži za uporabu i općenito učenje jer je njegova sintaksa, moglo bi se reći, kriptirana.
  			\item Dratex paket je vrlo mali paket za izradu grafike. U usporedbi sa TikZ-om i ostalim paketima je vrlo malen što može, a i ne mora biti njegova prednost.
  			\item Metapost je program koji je vrlo moćna alternativa TikZ-u. Prije je bio zaseban program što je prouzročilo mnogo problema, no sada je ugrađen unutar LaTeX-a.
  			\item Xfig program je važna alternativa TikZ-u za korisnike koji ne žele "programirati" svoju grafiku kao što je to potrebno uz TikZ i ostale navedene pakete. Postoji program koji pretvara xfig grafiku u TikZ.
  			\end{itemize}
  		\end{itemize}	
	\end{frame}	

\end{document}
 
